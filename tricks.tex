\documentclass[10pt,a4paper]{article}
\usepackage[utf8]{inputenc}

\usepackage{palatino}
\usepackage[english]{babel}
\usepackage{amsmath}
\usepackage{amsfonts}
\usepackage{amssymb}
\usepackage{graphicx}
\usepackage[left=2cm,right=2cm,top=2cm,bottom=2cm]{geometry}

\title{\textsc{Calculus Tricks and Traps}}
\date{}
\author{Schrödinger's Cat}

\begin{document}
\maketitle
\newpage

\begin{itemize}
\item Is it true that if a function has a vertical asymptote, its derivative has a vertical asymptote at the same place?
\item Is it true that if a function is continuous on the interval $(a,b)$ and its graph is a smooth curve (no sharp corners) on that interval, then the function is differentiable at any point on $(a, b)$?
\item Is it true that if  a function is not monotone, then it does not have an inverse function?
\item If a function $f$ is continuous for all real $x$ and $\lim_{n \to \infty} f(n) = A$ for natural numbers $n$, then $\lim_{x \to \infty} f(x) = A$. True or false?
\item If a function is defined on $\mathbb{R}^2$, it cannot be continuous at only one point. Discuss.
\item Discuss the following statement: If $f(x,y)$ approaches zero, as a point approaches the origin along any algebraic curve $y = cx^{m/n}$ , where $c \in \mathbb{R}$ and $m,n \in \mathbb{N}$ ($x \geq 0$ in case the exponential fraction is in lowest terms and $n$ is even), then the limit of $f(x,y)$ is zero at the origin.
\end{itemize}

\end{document}